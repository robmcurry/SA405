\documentclass[11pt]{article}

%% MinionPro fonts 
%\usepackage[lf]{MinionPro}
%\usepackage{MnSymbol}
\usepackage{microtype}

%% Margins
\usepackage{geometry}
\geometry{verbose,letterpaper,tmargin=1in,bmargin=1in,lmargin=1in,rmargin=1in}

%% Other packages
\usepackage{amsmath}
\usepackage{amsthm}
\usepackage{amsfonts}
\usepackage{mathrsfs}
\usepackage[shortlabels]{enumitem}
\usepackage{titlesec}
\usepackage{soul}
\usepackage{tikz}
\usepackage{mathtools}
\usepackage{pgfplots}
\usepackage{tikz-3dplot}
\usepackage{algorithmic}
\usepackage[export]{adjustbox}
\usepackage{tcolorbox}

%% Paragraph style settings
\setlength{\parskip}{\medskipamount}
\setlength{\parindent}{0pt}

%% Change itemize bullets
\renewcommand{\labelitemi}{$\bullet$}
\renewcommand{\labelitemii}{$\circ$}
\renewcommand{\labelitemiii}{$\diamond$}
\renewcommand{\labelitemiv}{$\cdot$}

%% Colors
\definecolor{rred}{RGB}{204,0,0}
\definecolor{ggreen}{RGB}{0,145,0}
\definecolor{yyellow}{RGB}{255,185,0}
\definecolor{bblue}{rgb}{0.2,0.2,0.7}
\definecolor{ggray}{RGB}{190,190,190}
\definecolor{ppurple}{RGB}{160,32,240}
\definecolor{oorange}{RGB}{255,165,0}

%% Shrink section fonts
\titleformat*{\section}{\normalsize\bf}
\titleformat*{\subsection}{\normalsize\bf}
\titleformat*{\subsubsection}{\normalsize\it}

% %% Compress the spacing around section titles
\titlespacing*{\section}{0pt}{1.5ex}{0.75ex}
\titlespacing*{\subsection}{0pt}{1ex}{0.5ex}
\titlespacing*{\subsubsection}{0pt}{1ex}{0.5ex}

%% amsthm settings
\theoremstyle{definition}
\newtheorem{problem}{Problem}
\newtheorem{example}{Example}
\newtheorem*{theorem}{Theorem}
\newtheorem*{bigthm}{Big Theorem}
\newtheorem*{biggerthm}{Bigger Theorem}
\newtheorem*{bigcor1}{Big Corollary 1}
\newtheorem*{bigcor2}{Big Corollary 2}

%% tikz settings
\usetikzlibrary{calc}
\usetikzlibrary{patterns}
\usetikzlibrary{decorations}
\usepgfplotslibrary{polar}

%% algorithmic setup
\algsetup{linenodelimiter=}
\renewcommand{\algorithmiccomment}[1]{\quad// #1}
\renewcommand{\algorithmicrequire}{\emph{Input:}}
\renewcommand{\algorithmicensure}{\emph{Output:}}

%% Answer box macros
%% \answerbox{alignment}{width}{height}
\newcommand{\answerbox}[3]{%
  \fbox{%
    \begin{minipage}[#1]{#2}
      \hfill\vspace{#3}
    \end{minipage}
  }
}

%% \answerboxfull{alignment}{height}
\newcommand{\answerboxfull}[2]{%
  \answerbox{#1}{6.38in}{#2} 
}

%% \answerboxone{alignment}{height} -- for first-level bullet
\newcommand{\answerboxone}[2]{%
  \answerbox{#1}{6.0in}{#2} 
}

%% \answerboxtwo{alignment}{height} -- for second-level bullet
\newcommand{\answerboxtwo}[2]{%
  \answerbox{#1}{5.8in}{#2}
}

%% special boxes
\newcommand{\wordbox}{\answerbox{c}{1.2in}{.5cm}}
\newcommand{\catbox}{\answerbox{c}{.5in}{.7cm}}
\newcommand{\letterbox}{\answerbox{c}{.7cm}{.5cm}}

%% Miscellaneous macros
\newcommand{\tstack}[1]{\begin{multlined}[t] #1 \end{multlined}}
\newcommand{\cstack}[1]{\begin{multlined}[c] #1 \end{multlined}}
\newcommand{\ccite}[1]{\only<presentation>{{\scriptsize\color{gray} #1}}\only<article>{{\small [#1]}}}
\newcommand{\grad}{\nabla}
\newcommand{\ra}{\ensuremath{\rightarrow}~}
\newcommand{\maximize}{\text{maximize}}
\newcommand{\minimize}{\text{minimize}}
\newcommand{\subjectto}{\text{subject to}}
\newcommand{\trans}{\mathsf{T}}
\newcommand{\bb}{\mathbf{b}}
\newcommand{\bx}{\mathbf{x}}
\newcommand{\bc}{\mathbf{c}}
\newcommand{\bd}{\mathbf{d}}

%% LP format
%    \begin{align*}
%      \maximize \quad & \mathbf{c}^{\trans} \mathbf{x}\\
%      \subjectto \quad & A \mathbf{x} = \mathbf{b}\\
%                       & \mathbf{x} \ge \mathbf{0}
%    \end{align*}


%% Redefine maketitle
\makeatletter
\renewcommand{\maketitle}{
  \noindent SA405 -- AMP \hfill Rader \S 3.2 \\

  \begin{center}\Large{\textbf{\@title}}\end{center}
}
\makeatother

%% ----- Begin document ----- %%
\begin{document}
  
\title{Lesson 9 Supplement:  Excel and Multiple Optimal Solutions}

\maketitle

%%%
\section{Today...}

\begin{itemize}
	\item  read and write to Excel from Python using pandas, xlrd, xlwrt
	\item  test for multiple optimal soluitons
\end{itemize}

\section{Main lesson}
The main lesson for today is in a Jupyter notebook.  We will just collect some useful ideas here for easy access.

\section{pandas: commonly used commands}



\begin{itemize}
\item  Import pandas:  \begin{center} \texttt{import pandas as pd} \end{center}
\item  Import numpy:  \begin{center} \texttt{import numpy as np} \end{center}

\item Import xlrd: \begin{center} \texttt{import xlrd}  \end{center}
\item Import xlwrt: \begin{center} \texttt{import xlwrt}  \end{center}

\item Import XlsxWriter: \begin{center} \texttt{import xlsxwriter}  \end{center}

\item  Open an existing Excel file called ``FileName.xlsx'': \begin{center} \texttt{df = pd.read\_excel ('FileName.xlsx')} \end{center}
\emph{Note: This only works if the Python file (Jupyter notebook) and the Excel file are in the same directory.  Otherwise, we need the full path to the Excel file in quotes.}

\emph{Another Note: pandas will always assign labels to rows and columns, so you that you reference the number row/column or the name of that row/column. It will automatically assign a numeric label of zero. If you have a better way to label columns, then you can set `index\_col=value' to some column value.}


%\emph{Quote: By default, pandas will automatically assign a numeric index or row label starting with zero. You may want to leave the default index as such if your data doesn’t have a column with unique values that can serve as a better index. In case there is a column that you feel would serve as a better index, you can override the default behavior by setting index\_col property to a column. It takes a numeric value for setting a single column as index or a list of numeric values for creating a multi-index.}

\item Open a specific sheet: \begin{center} \texttt{df = pd.read\_excel(r'File\_name.xlsx', sheet\_name='Sheet\_Name')}\end{center}

\item Importingonly a subset of columns: \begin{center}\texttt{data = pd.read\_excel ('FileName.xlsx') }


\texttt{df = pd.DataFrame(data, columns= ['ColumnName'])}
\end{center}
%\item  Open a sheet called ``my\_sheet'' in the Workbook called ``wb''  \begin{center} \texttt{sh = wb.sheets(`my\_sheet')} \end{center}
%\item  Add a sheet called ``my\_sheet'' in the Workbook called ``wb'' \begin{center} \texttt{sh = wb.sheets.add(`my\_sheet')} \end{center}

%How to add a sheet

%\item  To run the previous two commands without errors, we need to know whether or not the sheet ``my\_sheet'' exists in Workbook ``wb''.  This code allows me to try to add the sheet, and if that doesn't the sheet already exists and I can just open it.
%\begin{center}
%\begin{verbatim}
%try:                                      # try to add sheet
 %   sh = wb.sheets.add(`my_sheet')   
%except:                                   # otherwise, open the sheet
 %   sh = wb.sheets(`my_sheet')     
%\end{verbatim}
%\end{center}

\newpage

%Other stuff:

\item If you want to combine multiple sheets of data from 'Sheet\_Name\_1', 'Sheet\_Name\_3', 'Sheet\_Name\_3':

\begin{center}
\texttt{df = pd.read\_excel(r'File\_name.xlsx', sheet\_name='Sheet\_Name\_1')}

\texttt{df2 = pd.read\_excel(r'File\_name.xlsx', sheet\_name='Sheet\_Name\_2')}

\texttt{df3 = pd.read\_excel(r'File\_name.xlsx', sheet\_name='Sheet\_Name\_3')}

\texttt{combined\_values = pd.concat([df,df2,df3,])}
\end{center}


\item Exporting a DataFrame to Excel:

\begin{center}

\texttt{df.to\_excel('output.xlsx')}

\end{center}

\end{itemize}

\section{Other Python reminders}
\begin{itemize}
\item We can use a \texttt{range()} object in \texttt{for} loops.  For example,
\begin{center} \texttt{for i in range(3):}  \end{center}
loops over $i = 0, 1, 2$.
\item  Recall that list indexing always starts at 0.
\item To index over the list of lists,
\begin{center} \texttt{mylist = [[a,b,c],[d,e,f]]}, \end{center}
the item at position 
\begin{center} \texttt{mylist[0][1]} \end{center}
is b (list 0, position 1).
\end{itemize}


\section{Multiple Optimal Solutions}

Sometimes we might want to explore different optimal solutions to the same integer or linear program.  This means, a solution that has the same objective value, but different values for the variables.  In order to accomplish this, we can take the following steps:
\begin{enumerate}
\item Solve the model the first time
\item Add a constraint to the model that prevents the same optimal solution from occuring
\begin{itemize}
\item We have to be careful not to make the constraint too restrictive, or it may prevent other optimal solutions from occuring.
\end{itemize}
\item Solve the model again and see what happens
\begin{itemize}
\item If we get another optimal solution with the same objective value, then the model has multiple optimal solutions.  We can add yet another constraint to check for a third optimal solution, etc.
\item If we get infeasible or an optimal solution with a worse objective value, then there was only the one optimal solution.
\item Is it possible to add a constraint to the model and get a better better optimal objective value?
\end{itemize}
\end{enumerate}
\end{document}
