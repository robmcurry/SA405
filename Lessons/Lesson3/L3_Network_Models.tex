\documentclass[11pt]{article}

%% MinionPro fonts 
%\usepackage[lf]{MinionPro}
%\usepackage{MnSymbol}
\usepackage{microtype}

%% Margins
\usepackage{geometry}
\geometry{verbose,letterpaper,tmargin=1in,bmargin=1in,lmargin=1in,rmargin=1in}

%% Other packages
\usepackage{amsmath}
\usepackage{amsthm}
\usepackage[shortlabels]{enumitem}
\usepackage{titlesec}
\usepackage{soul}
\usepackage{tikz}
\usepackage{mathtools}
\usepackage{pgfplots}
\usepackage{tikz-3dplot}
\usepackage{algorithmic}
\usepackage[export]{adjustbox}
\usepackage{tcolorbox}

%% Get more space on the page!
%\addtolength{\topmargin}{-1cm}
%\addtolength{\textheight}{3cm}

%% Paragraph style settings
\setlength{\parskip}{\medskipamount}
\setlength{\parindent}{0pt}

%% Change itemize bullets
\renewcommand{\labelitemi}{$\bullet$}
\renewcommand{\labelitemii}{$\circ$}
\renewcommand{\labelitemiii}{$\diamond$}
\renewcommand{\labelitemiv}{$\cdot$}

%% Colors
\definecolor{rred}{RGB}{204,0,0}
\definecolor{ggreen}{RGB}{0,145,0}
\definecolor{yyellow}{RGB}{255,185,0}
\definecolor{bblue}{rgb}{0.2,0.2,0.7}
\definecolor{ggray}{RGB}{190,190,190}
\definecolor{ppurple}{RGB}{160,32,240}
\definecolor{oorange}{RGB}{255,165,0}

%% Shrink section fonts
\titleformat*{\section}{\normalsize\bf}
\titleformat*{\subsection}{\normalsize\bf}
\titleformat*{\subsubsection}{\normalsize\it}

% %% Compress the spacing around section titles
\titlespacing*{\section}{0pt}{1.5ex}{0.75ex}
\titlespacing*{\subsection}{0pt}{1ex}{0.5ex}
\titlespacing*{\subsubsection}{0pt}{1ex}{0.5ex}

%% amsthm settings
\theoremstyle{definition}
\newtheorem{problem}{Problem}
\newtheorem{example}{Example}
\newtheorem*{theorem}{Theorem}
\newtheorem*{bigthm}{Big Theorem}
\newtheorem*{biggerthm}{Bigger Theorem}
\newtheorem*{bigcor1}{Big Corollary 1}
\newtheorem*{bigcor2}{Big Corollary 2}

%% tikz settings
\usetikzlibrary{calc}
\usetikzlibrary{patterns}
\usetikzlibrary{decorations}
\usepgfplotslibrary{polar}

%% algorithmic setup
\algsetup{linenodelimiter=}
\renewcommand{\algorithmiccomment}[1]{\quad// #1}
\renewcommand{\algorithmicrequire}{\emph{Input:}}
\renewcommand{\algorithmicensure}{\emph{Output:}}

%% Answer box macros
%% \answerbox{alignment}{width}{height}
\newcommand{\answerbox}[3]{%
  \fbox{%
    \begin{minipage}[#1]{#2}
      \hfill\vspace{#3}
    \end{minipage}
  }
}

%% \answerboxfull{alignment}{height}
\newcommand{\answerboxfull}[2]{%
  \answerbox{#1}{6.38in}{#2} 
}

%% \answerboxone{alignment}{height} -- for first-level bullet
\newcommand{\answerboxone}[2]{%
  \answerbox{#1}{6.0in}{#2} 
}

%% \answerboxtwo{alignment}{height} -- for second-level bullet
\newcommand{\answerboxtwo}[2]{%
  \answerbox{#1}{5.8in}{#2}
}

%% special boxes
\newcommand{\wordbox}{\answerbox{c}{1.2in}{.7cm}}
\newcommand{\catbox}{\answerbox{c}{.5in}{.7cm}}
\newcommand{\letterbox}{\answerbox{c}{.7cm}{.7cm}}

%% Miscellaneous macros
\newcommand{\tstack}[1]{\begin{multlined}[t] #1 \end{multlined}}
\newcommand{\cstack}[1]{\begin{multlined}[c] #1 \end{multlined}}
\newcommand{\ccite}[1]{\only<presentation>{{\scriptsize\color{gray} #1}}\only<article>{{\small [#1]}}}
\newcommand{\grad}{\nabla}
\newcommand{\ra}{\ensuremath{\rightarrow}~}
\newcommand{\maximize}{\text{maximize}}
\newcommand{\minimize}{\text{minimize}}
\newcommand{\subjectto}{\text{subject to}}
\newcommand{\trans}{\mathsf{T}}
\newcommand{\bb}{\mathbf{b}}
\newcommand{\bx}{\mathbf{x}}
\newcommand{\bc}{\mathbf{c}}
\newcommand{\bd}{\mathbf{d}}

%% LP format
%    \begin{align*}
%      \maximize \quad & \mathbf{c}^{\trans} \mathbf{x}\\
%      \subjectto \quad & A \mathbf{x} = \mathbf{b}\\
%                       & \mathbf{x} \ge \mathbf{0}
%    \end{align*}


%% Redefine maketitle
\makeatletter
\renewcommand{\maketitle}{
  \noindent SA405 -- AMP \hfill Rader \S 2.9 \\

  \begin{center}\Large{\textbf{\@title}}\end{center}
}
\makeatother

%% ----- Begin document ----- %%
\begin{document}
  
\title{Lesson 3. Network Flows}


\maketitle
%\vspace{-1cm}
%\begin{center}\large{Transportation Models and  Minimum Cost Network Flows} \end{center}

%%%
\section{Today...}

\begin{itemize}
	\item  We model two \textbf{network flow problems}:
	\begin{itemize}
		\item a \textbf{transportation problem}, and
		\item a \textbf{minimum cost network flow problem}.
	\end{itemize}
\end{itemize}

%%%
\section{Transportation Model}
(Example 2.11, p.57) A local baked goods company has two bakeries where they bake their goods, which they then ship to three different area stores to sell.  Each bakery can produce up to 50 truckloads of baked goods per week, and each bakery can supply any of the stores.  The weekly demands (in truckloads) anticipated at each store along with the transportation costs (per truckload) are provided in the tables below.  Note that partial truckloads cost just as much as full truckloads.  How many truckloads should be sent from each bakery to each store in order to minimize total shipping cost?  
\begin{center}
\begin{tabular}{|c|c|}
\hline
 & Demand \\
\hline
Store 1 & 30 \\
\hline
Store 2 & 20 \\
\hline
Store 3 & 40 \\
\hline
\end{tabular}
\hspace{1cm}
\begin{tabular}{|c|c|c|c|}
\hline
& Store 1 & Store 2 & Store 3 \\
\hline
Bakery 1 & \$20 & \$45 & \$35 \\
\hline
Bakery 2 & \$35 & \$35 & \$50 \\
\hline
\end{tabular}
\end{center}


\subsection{Graphical representation of the network.}
\begin{itemize}
\item Bakeries and stores are represented by \emph{nodes} (or \emph{vertices}).  The bakeries are called \textbf{supply nodes}, and the stores are called \textbf{demand nodes}.  Label each node ($b1$, $b2$, $s1$, $s2$, $s3$).    Write the supply or demand amounts next to each node. 
\item \textbf{Directed arcs} (or \textbf{directed edges}) represent flow of goods.  Use an arrow on the demand end of each arc to indicate the direction of the flow of goods.  Label each arc with the cost per truckload.  
\end{itemize}

Draw a \textbf{directed graph} to represent the problem:

\newpage
Find a feasible (not necessarily optimal) solution, just to get a feel for the problem.  \vspace{1 in}


\subsection{Concrete model: transportation problem}
Formulate a concrete model that finds a feasible transportation strategy that minimizes cost:
\begin{itemize}

\item  Define decision variables.  
\vspace{1 in}

\item  Add the decision variables to the network diagram on page 1.

\item Describe objective and constraints.
\vspace{1.5in}
\item Write a concrete model for this problem using the decision variables defined above. 
\end{itemize}

\newpage
\subsection{Abstract model: tranportation problem}
Formulate an abstract model for the bakery problem. Clearly define sets, parameters, and variables.


\vfill
\subsection{Balancing supply and demand}
\begin{itemize}
\item When \emph{total supply} $>$ \emph{total demand}, we can model and solve the problem, as we saw in the bakery problem. When \emph{total supply} $<$ \emph{total demand}, the problem is \wordbox, but we still might want to solve the problem to meet as much demand as possible at a minimum cost.
\item When \emph{total supply} $=$ \emph{total demand}, we can use $=$ in the place of all inequalities in our constraints and we can write all of our constraints using the same format (we will see this soon), which is really nice.
\end{itemize}
\begin{tcolorbox}  We can always write a balanced problem with \emph{total supply} $=$ \emph{total demand} by adding a \textbf{dummy node} and \textbf{dummy arcs} with well-chosen parameter values (supply/demand and arc costs). 
\end{tcolorbox}

\newpage
%\begin{problem}
\textbf{Problem 1.}
\smallskip
\begin{enumerate}[(a)]
\item  In the original bakery problem, we had $100 = total~supply > total~demand = 90$.    We wish to add a dummy node and arcs so that supply and demand are balanced and that the original arc variables still provide the optimal solution to the original problem.
\begin{enumerate}[(i)]
\item Add a dummy node, along with its supply or demand.   Also add any dummy arcs entering or leaving the node, as well as their shipping costs and associated variables. 
\item In an optimal solution, what do the values of the variables associated with the dummy arcs tell us? 
\end{enumerate}  
\vfill

\item  Now suppose we have the original bakery problem, but now the demand at each store increases by 10, so $total~supply = 100$, and $total~demand = 120$.  We still want to know the minimum cost way to meet as much demand as possible.
\begin{enumerate}[(i)]
\item Add a dummy node, along with its supply or demand to accurately capture the logic of the problem.  Also add any dummy arcs entering or leaving the node, as well as their shipping costs and associated variables.
\item In an optimal solution, what do the values of the variables associated with the dummy arcs tell us?
\end{enumerate}  
\end{enumerate}
\vfill
%\end{problem}

\newpage
\section{Minimum cost network flow model}
(This is similar to Example 2.12, p 60) Consider the bakery again, only this time there are two warehouses that baked goods must be delivered to before they are taken to their final destinations.  The warehouses ($w1$ and $w2$) have no supply nor demand.  These are the transportation costs:

\begin{center}
\begin{tabular}{|c|c|c|c|}
\hline
& Warehouse 1 & Warehouse 2 \\
\hline
Bakery 1 & \$10 & \$15 \\
\hline
Bakery 2 & \$20 & \$25 \\
\hline
\end{tabular}
%\hspace{.5in}
\begin{tabular}{|c|c|c|c|}
\hline
& Store 1 & Store 2 & Store 3 \\
\hline
Warehouse 1 & \$20 & \$45 & \$35 \\
\hline
Warehouse 2 & \$35 & \$35 & \$50 \\
\hline
\end{tabular}
\end{center}
Also, assume that the store demands are $d_{s1} = 35$, $d_{s2} = 25$, and $d_{s3} = 40$, so that $total~supply = total~demand = 100$.
\begin{itemize}
\item The warehouses are called \textbf{transshipment nodes}. For the sake of simplicity, I'll refer to these as \emph{relay} nodes.  
\item This is called a \textbf{transshipment problem} or a \textbf{minimum cost network flow problem}.
\end{itemize}

Draw the new network diagram.  It is standard practice to combine supply and demand at each node into a single parameter, $b = supply - demand$.  Add labels for $b = supply - demand$ for each node, as well as all arc costs. 
\vfill 


\subsection{Concrete model: minimum cost network flow}
Write a concrete model for the bakery problem with warehouses.
\vfill \vfill

\newpage
\subsection{Balance of flow constraints}
Before writing the abstract model, it will be helpful to rearrange the constraints so that they all have the same form.  First, verify that you have \textbf{one constraint per node}, and that every constraints as one of the following  forms:
\[
supply ~=~ flow~out  ~~~~~\text{or}~~~~~  flow~in ~=~ demand ~~~~~\text{or}~~~~~  flow~in ~=~ flow~out.
\]
We can write all of our constraints using the same form by adding the first two forms above:
\begin{tcolorbox}
\[
supply + flow~in ~=~ demand + flow~out.^*
\]
These are called \textbf{flow-balance} constraints.
\end{tcolorbox}
$^*$Note that the validity of these constraints require that \emph{total supply = total demand} in the problem. 

\begin{itemize}
\item Why does it make sense to combine $supply$ with $flow~out$ and $demand$ with $flow~in$ in the balance of flow constraints?
\vfill

\item Write a general form for the \textbf{flow-balance} using $b = supply - demand$:

\end{itemize}
\vfill
\begin{tcolorbox}
\textbf{Balance of flow:} 
\begin{center}
 $ \hspace{1.5in}  \hspace{1.5in} = ~ b$
 \end{center}
 Must be included for \emph{every node} in the network.
\end{tcolorbox}

\newpage


\subsection{Abstract model: minimum cost network flow}
\begin{itemize}
\item  Write the abstract model for this minimum cost network flow problem.  
\vfill \vfill
\end{itemize}

\end{document}
