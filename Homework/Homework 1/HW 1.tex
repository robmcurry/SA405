\documentclass[11pt]{article}

%% MinionPro fonts 
%\usepackage[lf]{MinionPro}
%\usepackage{MnSymbol}
\usepackage{microtype}

%% Margins
\usepackage{geometry}
\geometry{verbose,letterpaper,tmargin=1in,bmargin=1in,lmargin=1in,rmargin=1in}

%% Other packages
\usepackage{amsmath}
\usepackage{amsthm}
\usepackage[shortlabels]{enumitem}
\usepackage{titlesec}
\usepackage{soul}
\usepackage{tikz}
\usepackage{mathtools}
\usepackage{pgfplots}
\usepackage{tikz-3dplot}
\usepackage{algorithmic}
\usepackage{optprog}
\usepackage[export]{adjustbox}
\usepackage{tcolorbox}

%% Paragraph style settings
\setlength{\parskip}{\medskipamount}
\setlength{\parindent}{0pt}

%% Change itemize bullets
\renewcommand{\labelitemi}{$\bullet$}
\renewcommand{\labelitemii}{$\circ$}
\renewcommand{\labelitemiii}{$\diamond$}
\renewcommand{\labelitemiv}{$\cdot$}

%% Colors
\definecolor{rred}{RGB}{204,0,0}
\definecolor{ggreen}{RGB}{0,145,0}
\definecolor{yyellow}{RGB}{255,185,0}
\definecolor{bblue}{rgb}{0.2,0.2,0.7}
\definecolor{ggray}{RGB}{190,190,190}
\definecolor{ppurple}{RGB}{160,32,240}
\definecolor{oorange}{RGB}{255,165,0}

%% Shrink section fonts
\titleformat*{\section}{\normalsize\bf}
\titleformat*{\subsection}{\normalsize\bf}
\titleformat*{\subsubsection}{\normalsize\it}

% %% Compress the spacing around section titles
\titlespacing*{\section}{0pt}{1.5ex}{0.75ex}
\titlespacing*{\subsection}{0pt}{1ex}{0.5ex}
\titlespacing*{\subsubsection}{0pt}{1ex}{0.5ex}

%% amsthm settings
\theoremstyle{definition}
\newtheorem{problem}{Problem}
\newtheorem{example}{Example}
\newtheorem*{theorem}{Theorem}
\newtheorem*{bigthm}{Big Theorem}
\newtheorem*{biggerthm}{Bigger Theorem}
\newtheorem*{bigcor1}{Big Corollary 1}
\newtheorem*{bigcor2}{Big Corollary 2}

%% tikz settings
\usetikzlibrary{calc}
\usetikzlibrary{patterns}
\usetikzlibrary{decorations}
\usepgfplotslibrary{polar}

%% algorithmic setup
\algsetup{linenodelimiter=}
\renewcommand{\algorithmiccomment}[1]{\quad// #1}
\renewcommand{\algorithmicrequire}{\emph{Input:}}
\renewcommand{\algorithmicensure}{\emph{Output:}}

%% Answer box macros
%% \answerbox{alignment}{width}{height}
\newcommand{\answerbox}[3]{%
  \fbox{%
    \begin{minipage}[#1]{#2}
      \hfill\vspace{#3}
    \end{minipage}
  }
}

%% \answerboxfull{alignment}{height}
\newcommand{\answerboxfull}[2]{%
  \answerbox{#1}{6.38in}{#2} 
}

%% \answerboxone{alignment}{height} -- for first-level bullet
\newcommand{\answerboxone}[2]{%
  \answerbox{#1}{6.0in}{#2} 
}

%% \answerboxtwo{alignment}{height} -- for second-level bullet
\newcommand{\answerboxtwo}[2]{%
  \answerbox{#1}{5.8in}{#2}
}

%% special boxes
\newcommand{\wordbox}{\answerbox{c}{1.2in}{.7cm}}
\newcommand{\catbox}{\answerbox{c}{.5in}{.7cm}}
\newcommand{\letterbox}{\answerbox{c}{.7cm}{.7cm}}

%% Miscellaneous macros
\newcommand{\tstack}[1]{\begin{multlined}[t] #1 \end{multlined}}
\newcommand{\cstack}[1]{\begin{multlined}[c] #1 \end{multlined}}
\newcommand{\ccite}[1]{\only<presentation>{{\scriptsize\color{gray} #1}}\only<article>{{\small [#1]}}}
\newcommand{\grad}{\nabla}
\newcommand{\ra}{\ensuremath{\rightarrow}~}
\newcommand{\maximize}{\text{maximize}}
\newcommand{\minimize}{\text{minimize}}
\newcommand{\subjectto}{\text{subject to}}
\newcommand{\trans}{\mathsf{T}}
\newcommand{\bb}{\mathbf{b}}
\newcommand{\bx}{\mathbf{x}}
\newcommand{\bc}{\mathbf{c}}
\newcommand{\bd}{\mathbf{d}}

%% LP format
%    \begin{align*}
%      \maximize \quad & \mathbf{c}^{\trans} \mathbf{x}\\
%      \subjectto \quad & A \mathbf{x} = \mathbf{b}\\
%                       & \mathbf{x} \ge \mathbf{0}
%    \end{align*}


%% Redefine maketitle
\makeatletter
\renewcommand{\maketitle}{
  \noindent SA405 

  \begin{center}\Large{\textbf{\@title}}\end{center}
}
\makeatother

%% ----- Begin document ----- %%
\begin{document}
\title{HW1: LP Review and IP Formulation}


\maketitle

\textbf{1.} Giapetto’s Woodcarving, Inc., manufactures two types of wooden toys: soldiers and trains. A soldier sells for \$27 and uses \$10 worth of raw materials. Each solder that is manufactured increases Giapetto’s variable labor and overhead costs by \$14. A train sells for \$21 and uses \$9 worth of raw materials. Each train built increases Giapetto’s variable labor and overhead costs by \$10. The manufacture of wooden soldiers and trains requires two types of skilled labor: carpentry and finishing. A soldier requires 2 hours of finishing labor and 1 hour of carpentry labor. A train requires 1 hour of finishing and 1 hour of carpentry labor. Each week, Giapetto can obtain all the needed raw material but only 100 finishing hours and 80 carpentry hours. Demand for trains is unlimited, but at most 40 soldiers are bought each week. Giapetto wants to maximize weekly profit (revenue - costs). Formulate a concrete and parameterized LP model that can be used to maximize Giapetto’s weekly profit.

\vspace{1in}

\textbf{2.} The Concrete Guys makes two types of (dry) concrete mix using cement, sand, and gravel. The regular mix contains (exactly) 30\% of cement, 15\% of sand, and 55\% of gravel (by weight), and sells for 5 cent/lb. The extra-strong mix must contain at least 50\% of cement, at least 5\% of sand, and at least 20\% of gravel, and sells for 8 cent/lb. The Concrete Guys has 100, 000 lb of cement, 50, 000 lb of sand, and 100, 000 lb of gravel in its warehouse. Formulate an LP to determine the amount of each mix the Concrete Guys should make in order to maximize its revenue.

\vspace{1in}

\textbf{3.} Funding 'R Us is considering four different investments: Investment 1 yields a net present value (NPV) of \$16,000; investment 2, an NPV of \$22,000; investment 3, an NPV of \$12,000; and investment 4, an NPV of \$8,000. Each investment requires a certain cash outflow at the present time: investment 1, \$5,000; investment 2, \$7,000; investment 3, \$4,000; and investment 4, \$3,000. Currently, \$14,000 is available for investment. Formulate a concrete and parameterized integer programming model whose solution will tell Funding 'R Us how to maximize the NPV obtained from investments 1--4. (\emph{Hint:} You can only decide whether to invest in an investment or not. What type of variable should you use?)  









\end{document}
