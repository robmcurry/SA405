\documentclass[11pt]{article}

%% MinionPro fonts 
%\usepackage[lf]{MinionPro}
%\usepackage{MnSymbol}
\usepackage{microtype}

%% Margins
\usepackage{geometry}
\geometry{verbose,letterpaper,tmargin=1in,bmargin=1in,lmargin=1in,rmargin=1in}

%% Other packages
\usepackage{amsmath}
\usepackage{amsthm}
\usepackage[shortlabels]{enumitem}
\usepackage{titlesec}
\usepackage{soul}
\usepackage{tikz}
\usepackage{mathtools}
\usepackage{pgfplots}
\usepackage{tikz-3dplot}
\usepackage{algorithmic}
\usepackage[export]{adjustbox}
\usepackage{tcolorbox}

%% Paragraph style settings
\setlength{\parskip}{\medskipamount}
\setlength{\parindent}{0pt}

%% Change itemize bullets
\renewcommand{\labelitemi}{$\bullet$}
\renewcommand{\labelitemii}{$\circ$}
\renewcommand{\labelitemiii}{$\diamond$}
\renewcommand{\labelitemiv}{$\cdot$}

%% Colors
\definecolor{rred}{RGB}{204,0,0}
\definecolor{ggreen}{RGB}{0,145,0}
\definecolor{yyellow}{RGB}{255,185,0}
\definecolor{bblue}{rgb}{0.2,0.2,0.7}
\definecolor{ggray}{RGB}{190,190,190}
\definecolor{ppurple}{RGB}{160,32,240}
\definecolor{oorange}{RGB}{255,165,0}

%% Shrink section fonts
\titleformat*{\section}{\normalsize\bf}
\titleformat*{\subsection}{\normalsize\bf}
\titleformat*{\subsubsection}{\normalsize\it}

% %% Compress the spacing around section titles
\titlespacing*{\section}{0pt}{1.5ex}{0.75ex}
\titlespacing*{\subsection}{0pt}{1ex}{0.5ex}
\titlespacing*{\subsubsection}{0pt}{1ex}{0.5ex}

%% amsthm settings
\theoremstyle{definition}
\newtheorem{problem}{Problem}
\newtheorem{example}{Example}
\newtheorem*{theorem}{Theorem}
\newtheorem*{bigthm}{Big Theorem}
\newtheorem*{biggerthm}{Bigger Theorem}
\newtheorem*{bigcor1}{Big Corollary 1}
\newtheorem*{bigcor2}{Big Corollary 2}

%% tikz settings
\usetikzlibrary{calc}
\usetikzlibrary{patterns}
\usetikzlibrary{decorations}
\usepgfplotslibrary{polar}

%% algorithmic setup
\algsetup{linenodelimiter=}
\renewcommand{\algorithmiccomment}[1]{\quad// #1}
\renewcommand{\algorithmicrequire}{\emph{Input:}}
\renewcommand{\algorithmicensure}{\emph{Output:}}

%% Answer box macros
%% \answerbox{alignment}{width}{height}
\newcommand{\answerbox}[3]{%
  \fbox{%
    \begin{minipage}[#1]{#2}
      \hfill\vspace{#3}
    \end{minipage}
  }
}

%% \answerboxfull{alignment}{height}
\newcommand{\answerboxfull}[2]{%
  \answerbox{#1}{6.38in}{#2} 
}

%% \answerboxone{alignment}{height} -- for first-level bullet
\newcommand{\answerboxone}[2]{%
  \answerbox{#1}{6.0in}{#2} 
}

%% \answerboxtwo{alignment}{height} -- for second-level bullet
\newcommand{\answerboxtwo}[2]{%
  \answerbox{#1}{5.8in}{#2}
}

%% special boxes
\newcommand{\wordbox}{\answerbox{c}{1.2in}{.5cm}}
\newcommand{\catbox}{\answerbox{c}{.5in}{.7cm}}
\newcommand{\letterbox}{\answerbox{c}{.7cm}{.5cm}}

%% Miscellaneous macros
\newcommand{\tstack}[1]{\begin{multlined}[t] #1 \end{multlined}}
\newcommand{\cstack}[1]{\begin{multlined}[c] #1 \end{multlined}}
\newcommand{\ccite}[1]{\only<presentation>{{\scriptsize\color{gray} #1}}\only<article>{{\small [#1]}}}
\newcommand{\grad}{\nabla}
\newcommand{\ra}{\ensuremath{\rightarrow}~}
\newcommand{\maximize}{\text{maximize}}
\newcommand{\minimize}{\text{minimize}}
\newcommand{\subjectto}{\text{subject to}}
\newcommand{\trans}{\mathsf{T}}
\newcommand{\bb}{\mathbf{b}}
\newcommand{\bx}{\mathbf{x}}
\newcommand{\bc}{\mathbf{c}}
\newcommand{\bd}{\mathbf{d}}


%% Redefine maketitle
\makeatletter
\renewcommand{\maketitle}{
  \noindent SA405 -- AMP \hfill Lesson \#9\\

  \begin{center}\Large{\textbf{\@title}}\end{center}
}
\makeatother

%% ----- Begin document ----- %%
\begin{document}
  
\title{Practice Problem \#9: Fixed-Charge Problem}

\maketitle

%%%
\section{The Sabre Problem}
In 2015, the printer manufacturer Sabre has set up shop creating printers,
copiers, and scanners to celebrate its 10th anniversary. Their machinist makes the
designs each machine out of plastic and metal. To begin manufacturing each machine, Sabre must pay a significant set up cost. All relevant data can be found in the table below.

\begin{center}
\begin{tabular}{|c|c|c|c|c|} \hline
 & Printers & Copiers & Scanners & Availability\\ \hline
Machinist Labor (Days) & 2 & 4 & 5 & 100 \\ \hline
Plastic (pounds) & 1& 1.5 & 1.8 & 30 \\ \hline
Metal (pounds) & 1& 1.5 & 1.8 & 30 \\ \hline
Profit (\$ per machine) & 52 & 30 & 20 &-- \\ \hline
Set-up Costs (\$) & 500 & 400 & 300  & -- \\ \hline
\end{tabular}
\end{center}
\subsection{Concrete Model}
Formulate Sabre's problem as a \textbf{concrete} integer programming model to maximize the total amount of profit. Define and describe all restrictions, the objective, and all decision variable(s).
\pagebreak

\subsection{Parameterized Model}
Sabre has come back to you for help. They found your original model to be super useful! They want to quickly scale up their operation. They want to know how to solve their problem with $N$ different types of machines and $R$ different types of resources (e.g., labor, metal, plastic). Formulate Sabre's problem as an \textbf{parameterized} integer programming model to maximize the total amount of profit. Define and describe all sets, parameters, and all decision variable(s).
\newpage
\section{Set-covering Models}
\subsection{Question:} Briefly describe the difference between set-covering, -packing, and -partitioning constraints.
\pagebreak
\subsection{Problem:}
USNA is organizing a yard-wide athletic decathlon. Assume you have a set of athletes in your company: Jamie, Daphne, Gary, and Jackie. Each athlete excels at least one sport. Jamie swims and plays basketball. Daphne plays squash and soccer. Gary plays basketball, squash, and croquet. Finally, Jackie swims, and plays basketball and squash. Your job is to hire a \#squad of athletes. You are given two requirements: (i)
there has to be at least one person on the team who plays each sport (i.e., swimming, basketball, soccer, squash, and croquet), and
(ii) your team should be as small as possible (maybe your team is running on a tight budget).

Use a network diagram to better visualize this problem.
\begin{enumerate}[a.]
\item Formulate this problem as a \textbf{concrete} integer programming model. Clearly define and describe all restrictions, the objective, and all decision variable(s).
\pagebreak

\item Formulate this problem as an \textbf{parameterized} integer programming model. Clearly define and describe all restrictions, the objective, and all decision variable(s).
\end{enumerate}



\end{document}
