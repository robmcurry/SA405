% LaTeX Article Template
\documentclass[12pt]{article}
%% Other packages
\usepackage{amsmath}
\usepackage{amsthm}
\usepackage{titlesec}
\usepackage{soul}
\usepackage{tikz}
\usepackage{tikz-3dplot}
\usepackage{amssymb}
\usepackage{multicol}
\usepackage{float}
\usepackage{calc}
\usepackage{fancybox}
\usepackage{array}
\usepackage[shortlabels]{enumitem}
\usepackage{framed}
\usepackage{hyperref}
\usepackage{longtable}

\newcolumntype{L}[1]{>{\raggedright\let\newline\\\arraybackslash\hspace{0pt}}m{#1}}
\newcolumntype{C}[1]{>{\centering\let\newline\\\arraybackslash\hspace{0pt}}m{#1}}
\newcolumntype{R}[1]{>{\raggedleft\let\newline\\\arraybackslash\hspace{0pt}}m{#1}}


%% Margins
\usepackage{geometry}
\geometry{verbose,letterpaper,tmargin=1in,bmargin=1in,lmargin=1in,rmargin=1in}

\newcommand{\menuchoice}[2]{{\ttfamily#1..#2}}
\newcommand{\dotdot}{..}

\usepackage{graphicx}

% Array vertical and horizontal stretch
% \def\arraystretch{1.5}%  1 is the default, change whatever you need
% \setlength{\tabcolsep}{12pt}

%\graphicspath{%
\graphicspath{{./figs/}}

%% Paragraph style settings
\setlength{\parskip}{\medskipamount}
\setlength{\parindent}{0pt}

%% Change itemize bullets
\renewcommand{\labelitemi}{$\bullet$}
\renewcommand{\labelitemii}{$\circ$}
\renewcommand{\labelitemiii}{$\diamond$}
\renewcommand{\labelitemiv}{$\cdot$}

%% Shrink section fonts
\titleformat*{\section}{\large\bf}
\titleformat*{\subsection}{\normalsize\it}
\titleformat*{\subsubsection}{\normalsize\bf}

% %% Compress the spacing around section titles
\titlespacing*{\section}{0pt}{1.5ex}{0.75ex}
\titlespacing*{\subsection}{0pt}{1ex}{0.5ex}
\titlespacing*{\subsubsection}{0pt}{1ex}{0.5ex}

%% amsthm settings
\theoremstyle{definition}
\newtheorem{problem}{Problem}
\newtheorem{example}{Example}
\newtheorem{mydef}{Definition}

%% Answer box macros
%% \answerbox{alignment}{width}{height}
\newcommand{\answerbox}[3]{%
  \fbox{%
    \begin{minipage}[#1]{#2}
      \hfill\vspace{#3}
    \end{minipage}
  }
}

%% \answerboxfull{alignment}{height}
\newcommand{\answerboxfull}[2]{%
  \answerbox{#1}{\textwidth}{#2} 
}

%% \answerboxone{alignment}{height} -- for first-level bullet
\newcommand{\answerboxone}[2]{%
  \answerbox{#1}{6.15in}{#2} 
}

%% \answerboxtwo{alignment}{height} -- for second-level bullet
\newcommand{\answerboxtwo}[2]{%
  \answerbox{#1}{5.8in}{#2}
}

%% \graphbox{xmin}{xmax}{ymin}{ymax}{scale}
\newcommand{\graphbox}[5]%[-5, 5, -5, 5, 0.33]
{
\begin{tikzpicture}
     [>=latex,scale=#5]
     
     % Coordinate axes
     \draw [->,very thick] (#1, 0) -- (#2, 0) node[right] {$x$};
     \draw [->,very thick] (0, #3) -- (0, #4) node[above] {$y$};
     
     % Grid
     \draw[step=1cm,thick,dotted] (#1,#3) grid (#2,#4);
   \end{tikzpicture}
   }


%% Redefine maketitle
\makeatletter
\renewcommand{\maketitle}{
  \noindent SA405 -- Advanced Mathematical Programming \hfill Fall 2021

  \begin{center}\large{\textbf{\@title}}\end{center}
}
\makeatother

% Set the beginning of a LaTeX document
\begin{document}

%\graphbox{-10}{3}{-5}{10}

\title{Syllabus}

%\graphbox[10][10]

\maketitle

\noindent \textbf{Course coordinator:}  Asst. Prof. Robert M. Curry  (rcurry@usna.edu)

\noindent \textbf{Textbook:}  \emph{Deterministic Operations Research}, by David Rader.

\noindent \textbf{Course description: }This course covers a range of advanced topics in mathematical programming. Topics include integer programming modeling, branch-and-bound methods, integer programming theory,  and algorithms. Students will also learn to use a set-based modeling language for an advanced integer programming solver. Topics may vary with instructor.

\noindent \textbf{Course objectives:}  By the end of this course, students will be able to
\vspace{-2mm}
\begin{enumerate}[(i)]
\item creatively and critically problem solve;
\item successfully collaborate in groups;
\item identify, model, and solve (using software) a variety of real-world problems that can be formulated as integer linear programs;
\item use Microsoft Excel and Python to develop a spreadsheet interface for a linked optimization model;
\item understand the theoretical and computational difficulty of integer linear optimization, along with associated theoretical and algorithmic considerations and algorithms.
\end{enumerate}
\noindent \textbf{Approximate weekly course schedule:} 

\renewcommand\arraystretch{1.5}
\begin{longtable}{ll}
Week \hspace{.2in} & Topic \\
\hline 
1 & Mathematical optimization modeling and software review.  \\
2 & Network Models\\
3 & Shortest Path Models\\ 
4 & Maximum Flow Models\\
5 &  Fixed-charge Models, Set-covering, \& Logical Constraints \\
6 & Review \& \textit{Exam \#1} \\
7 & Python/Pyomo Resources\\
8 & Minimum Spanning Tree Problem \\
9 & Traveling Salesperson Problem (TSP) \\
10 & Vehicle Routing Problems \& Subtour-elimination Constraints. \\
11 & Facility Location Models \\
11 & Review \& \textit{Exam \#2}  \\
12 & Integrating Python, Pyomo, and Microsoft Excel \& Improving Integer Feasible Region Formulations \\
13 & IP Formulations\\
14 & Branch \& Bound \\
15 & Project work\\
16 & Project presentations and Review \\
\end{longtable}

\end{document}
